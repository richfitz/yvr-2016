% set smartparens-mode to disable that godawful automatic thing
\documentclass[20pt]{beamer}
\input{style}
\input{positioning}

\usepackage{fontawesome}
\newfontfamily{\FA}{FontAwesome}
\def\twitter{{\FA \faTwitter}}
\def\heart{{\FA \faHeart}}
\def\tickmark{{\FA \faCheck}}
\def\rarrow{{\FA \relsize{-1}{\faArrowRight}}}
\def\imagetop#1{\vtop{\null\hbox{#1}}}
\def\code{\texttt}
\newcommand{\hrefp}[2]{\href{#1://#2}{#2}}
\newcommand{\twitterhandle}[1]{\href{https://twitter.com/#1}{@#1}}
\newcommand{\ghrepo}[1]{\href{https://github.com/#1}{\tt github.com/#1}}
\newcommand{\twitterhandlesm}[1]{\relsize{-1}{\color{b-grey}\href{https://twitter.com/#1}{@#1}}}

\begin{document}

\setbeamercolor*{palette primary}{fg=white,bg=b-darkgrey}
\setbeamercolor*{titlelike}{parent=palette primary}
\setbeamercolor*{normal text}{parent=palette primary}
\setbeamercolor*{itemize}{parent=palette primary}
\color{white}
\setbeamertemplate{navigation symbols}{}
\setbeamercolor{itemize/enumerate body}{fg=white}
\setbeamercolor{enumerate item}{fg=white}
\setbeamertemplate{itemize item}{\raisebox{.33ex}{\footnotesize\color{white}$\blacktriangleright$}}

\begin{frame}
  \begin{tikzpicture}[remember picture, overlay]
    \node (A) at ($(current page.center) + (0, .17\paperheight)$) {
      \resizebox{.9\paperwidth}{!}{%
        \color{b-blue}\bf Tools for collaborative data use}
    };
    \node[anchor=north] at (A.south) {
      \resizebox{.9\paperwidth}{!}{%
        \color{b-green}and for users who do not want to care}
    };
    \color{white}
    \node[anchor=south west] (B) at
    ($(current page.south west) + (0.05\paperwidth, 0.05\paperwidth)$) {
      \color{b-grey}\twitterhandle{rgfitzjohn}
    };
    \node[anchor=south west] at (B.north west) {
      Rich FitzJohn
    };
  \end{tikzpicture}
\end{frame}

% Some background about me, the department, their skills and how we
% get to this point.

\begin{frame}
  \herohigh{\color{b-blue}Two related problems}
  \bottomhanginglow {
    \begin{minipage}{\textwidth}
      \begin{itemize}
      \item Data distribution \& versioning
      \item Sharing sensitive data
      \end{itemize}
    \end{minipage}}
\end{frame}

% * We love git
% * But it sucks for data
%   - git annex (wtf is going on there)
%   - git lfs (needs a server)
%   - many others
%
%   Version control of data is hard!  Dat will solve the problem but I
%   have been waiting for it do so for ages, and they're mostly
%   interested in massively multiuser tabular data, rather than things
%   that scientists use, such as trees, database dumps, shapefiles,
%   whatever.
%
%   Because git is so central to New Science it all gets a bit
%   derailed when you hit any data of real size.
% \begin{frame}
%   \herohigh{\color{b-pink}Data versioning}
%   \bottomhanginglow{%
%     \raisebox{-.08\paperheight}{\includegraphics[height=.2\paperheight]{diagrams/dropbox}}%
%     \hspace{.05\paperwidth}{\LARGE vs}\hspace{.05\paperwidth}%
%     \raisebox{-.08\paperheight}{\includegraphics[height=.2\paperheight]{pics/logos/git}}}
%   % TODO: Other logos needed here: dat, github, figshare
% \end{frame}

\begin{frame}
  \herohigh{\color{b-pink}Data versioning}
  \bottomhanginglow{%
    \raisebox{-.07\paperheight}{\includegraphics[height=.15\paperheight]{pics/logos/git}}%
    \hspace{.025\paperwidth}{\LARGE +}\hspace{.03\paperwidth}%
    \raisebox{-.07\paperheight}{%
      \includegraphics<1>[height=.15\paperheight]{pics/database_icon1}%
      \includegraphics<2>[height=.15\paperheight]{pics/database_icon3}%
      \includegraphics<3>[height=.15\paperheight]{pics/database_icon5}}%
    \hspace{.025\paperwidth}{\LARGE =}\hspace{.03\paperwidth}%
  }
\end{frame}

% The usual solution proposed here is "just put it on figshare" or
% some other service.  But this solves only one of the set of
% problems.
\begin{frame}
  \hero{\color{b-green}``Just put it on}
  \begin{tikzpicture}[remember picture, overlay]%
    \node[anchor=east,align=right,font=\Huge\bfseries] (A) at ($(current page.east) + (-0.15\paperwidth, -.05\paperheight)$) {%
      \only<1>{dropbox''}%
      \only<2>{figshare''}%
      \only<3>{google drive''}%
    };
  \end{tikzpicture}
\end{frame}

\begin{frame}
  \minionhigh{\color{b-pink}How do we\ldots}
  \bottomhanging {
    \begin{minipage}{\textwidth}
      \begin{itemize}
      \item store the potentially large data?
        \only<2>{{\color{b-green}\tickmark}}
      \item load the data into R?
      \item work offline / cache reads?
      \item deal with multiple versions?
      \item control access to the data?
      \end{itemize}
    \end{minipage}}
\end{frame}

\begin{frame}
  \hero{\color{b-blue} datastorr}
  \bottomhangingverylow{\ghrepo{richfitz/datastorr}}
\end{frame}

% Here, I should show how this works in practice.  The key ingredients
% are:
%
% Store data against gh releases.
\begin{frame}
  \minionhigh{\color{b-blue}Storage: \color{b-green}GitHub releases}
  % Include a screenshot here I think.  Probably of baad or datastorr
  \bottomhanginglow{
    \begin{minipage}{.94\textwidth}
      \color{white}\raggedright
      ``We don't limit the total size of your binary release files, nor
      the bandwidth used to deliver them. However, each individual
      file must be under {\color{b-pink}2GB} in size.''
    \end{minipage}}
  \bottomright{\tt help.github.com/articles/distributing-large-binaries}
\end{frame}

\begin{frame}
  \minionhigh{\color{b-blue} configuration}
  \bottomhanging{
    \includegraphics<1>[width=.8\textwidth]{snippets/datastorr-json}}
  \bottomright{\tt github.com/richfitz/datastorr.example}
\end{frame}

% At this point we'll skip over how things get into the releases and
% work with the data assuming that they're there.  I'll loop back to
% show how we put the data up there once the benefits of doing so are
% more obvious.

% I'd like a slightly better demo version here; something with a small
% dataset that changes over time.  To do this we should take a dataset
% like mtcars and edit it a bit so that daff can show the changes that
% happen in a nice way.  The example I have, using readRDS is a bit
% artificial.  I just need something that I can meaningfully version.
% For now, leave as-is.

% Look at the help for daff to see if there is a good example of
% changing tabular data there.
\begin{frame}
  \minionhigh{\color{b-blue}Transfer: \color{b-green}\texttt{curl}}
  \bottomhanging{
    \begin{minipage}{\textwidth}
      \includegraphics<1>[width=\textwidth]{snippets/datastorr-transfer-1}
      \includegraphics<2>[width=\textwidth]{snippets/datastorr-transfer-2}
    \end{minipage}}
  \bottomright{\ghrepo{jeroenooms/curl}}
\end{frame}

\begin{frame}
  \minionhigh{\color{b-blue}Caching: \color{b-green}\texttt{storr}}
  % Diagram with storr showing the caching layer(s)
  % Cloud -> Disk -> Environment -> R object
  \bottomright{\ghrepo{richfitz/storr}}
\end{frame}

% Versioning; show different tables, perhaps run with daff
%   https://github.com/edwindj/daff
\begin{frame}
  \minionhigh{\color{b-blue}Versioning:
    \color{b-green}\texttt{datastorr} \&
    \color{b-green}\texttt{storr}}
  \bottomhanging{
    \includegraphics<1>[width=\textwidth]{snippets/datastorr-version-1}
    \includegraphics<2>[width=\textwidth]{snippets/datastorr-version-2}
    \includegraphics<3>[width=\textwidth]{snippets/datastorr-version-3}
  }
  \bottomright{\ghrepo{richfitz/datastorr}}
\end{frame}

\begin{frame}
  \minionhigh{\color{b-blue}Access control}
  \begin{minipage}{1.0\linewidth}
    \begin{itemize}
    \item Private repos = private data {\color{b-pink}{\FA \faLock}}
    \item Access with OAuth or GitHub token {\color{b-pink}{\FA
          \faUnlock}}
    \item Private organisation repos, too
    \end{itemize}
  \end{minipage}
\end{frame}


% I may need to wind it back a bit here; these are all the possible
% features of the package, and I do want to emphasise that this is
% meant to work well with the simple csv-as-moving-target use-case.
% So perhaps the creation part should be really simple.  I should
% point at an entire workflow here; a document or something that walks
% through the whole thing end-to-end.
\begin{frame}
  \hero{Making releases}
\end{frame}

\begin{frame}
  \herohigh{Making releases}
  \bottomhanging{
    \begin{minipage}{1.0\linewidth}
      \begin{enumerate}
        \item Create a GitHub Repo
      \end{enumerate}
    \end{minipage}}
\end{frame}

% Show how creation works

% Perhaps offer an interactive bit

% Picture of a cute cat.

\end{document}

%%% Local Variables:
%%% mode: latex
%%% TeX-master: t
%%% TeX-engine: xetex
%%% End:
